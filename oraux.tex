\documentclass[french,a4paper, 12pt]{article}

\usepackage[utf8]{inputenc}
\usepackage[T1]{fontenc}
\usepackage{lmodern} 
\usepackage{xspace}
\usepackage[main=french]{babel}

\title{Oraux}
\author{Samuel Gallay}

\begin{document}
\maketitle

\emph{Oraux des ENS}

\vspace{5px}

Je suis arrivé à Paris dimanche en début d'après-midi, et j'ai fait une visite complète du 5ième arrondissement. Il fait relativement chaud pour Paris, surtout la nuit...

\vspace{5px}

Oral de Physique pour Lyon, Cachan et Rennes : une catastrophe...

45 minutes, pas de préparation : On considère un condensateur cylindrique de hauteur $L$ dirigé selon l'axe $e_z$ avec le bas du condensateur en $z = 0$, de rayons intérieur et extérieurs $r_1$ et $r_2$. On se place dans l'ARQS, la charge dépendant du temps, et on demande d'abord de calculer la puissance dissipée (ie le flux du vecteur de Poynting) à travers la surface $z = L$.

J'ai commencé par dire des choses intelligentes, mais l'examinateur s'est rendu compte que je ne comprennais rien à l'ARQS, donc j'ai eu le droit à de nombreuses questions de cours. D'abord qu'est-ce que le vecteur nabla ? Remontrez moi l'équation de D'Alembert pour le champ électrique. Qu'est-ce que l'ARQS? En fait j'ai confondu au début  les deux courants dans l'équation de Maxwell-Ampère. J'ai dû expliquer ce qui se passait physiquement dans un condensateur, et je n'y arrivais pas. L'examinateur, par ailleurs très sympathique, notait les bêtises que je disais, et me faisait me contredire pour que je comprenne ce qu'il se passait. Distances caractériqtiques de l'ARQS, etc...

Bon, je suis très déçu de moi sur ce coup là, d'habitude j'arrive à sauver les meubles pendant mes oraux de physique...

\end{document}